% usenix.sty - to be used with latex2e for USENIX.
% To use this style file, look at the template usenix_template.tex
%
% $Id: usenix.sty,v 1.2 2005/02/16 22:30:47 maniatis Exp $
%
% The following definitions are modifications of standard article.sty
% definitions, arranged to do a better job of matching the USENIX
% guidelines.
% It will automatically select two-column mode and the Times-Roman
% font.

%
% USENIX papers are two-column.
% Times-Roman font is nice if you can get it (requires NFSS,
% which is in latex2e.

\if@twocolumn\else\input twocolumn.sty\fi
\usepackage{mathptmx}  % times roman, including math (where possible)

%
% USENIX wants margins of: 1" sides, 1" bottom, and 1" top.
% 0.25" gutter between columns.
% Gives active areas of 6.5" x 9"
%
\setlength{\textheight}{9.0in}
\setlength{\columnsep}{0.25in}
\setlength{\textwidth}{6.50in}

\setlength{\topmargin}{0.0in}

\setlength{\headheight}{0.0in}

\setlength{\headsep}{0.0in}

% Usenix wants no page numbers for camera-ready papers, so that they can
% number them themselves.  But submitted papers should have page numbers
% for the reviewers' convenience.
% 
%
% \pagestyle{empty}

%
% Usenix titles are in 14-point bold type, with no date, and with no
% change in the empty page headers.  The whole author section is 12 point
% italic--- you must use {\rm } around the actual author names to get
% them in roman.
%
\def\maketitle{\par
 \begingroup
   \renewcommand\thefootnote{\fnsymbol{footnote}}%
   \def\@makefnmark{\hbox to\z@{$\m@th^{\@thefnmark}$\hss}}%
    \long\def\@makefntext##1{\parindent 1em\noindent
            \hbox to1.8em{\hss$\m@th^{\@thefnmark}$}##1}%
   \if@twocolumn
     \twocolumn[\@maketitle]%
     \else \newpage
     \global\@topnum\z@
     \@maketitle \fi\@thanks
 \endgroup
 \setcounter{footnote}{0}%
 \let\maketitle\relax
 \let\@maketitle\relax
 \gdef\@thanks{}\gdef\@author{}\gdef\@title{}\let\thanks\relax}

\def\@maketitle{\newpage
 \vbox to 2.5in{
 \vspace*{\fill}
 \vskip 2em
 \begin{center}%
  {\Large\bf \@title \par}%
  \vskip 0.375in minus 0.300in
  {\large\it
   \lineskip .5em
   \begin{tabular}[t]{c}\@author
   \end{tabular}\par}%
 \end{center}%
 \par
 \vspace*{\fill}
% \vskip 1.5em
 }
}

%
% The abstract is preceded by a 12-pt bold centered heading
\def\abstract{\begin{center}%
{\large\bf \abstractname\vspace{-.5em}\vspace{\z@}}%
\end{center}}
\def\endabstract{}

%
% Main section titles are 12-pt bold.  Others can be same or smaller.
%
\def\section{\@startsection {section}{1}{\z@}{-3.5ex plus-1ex minus
    -.2ex}{2.3ex plus.2ex}{\reset@font\large\bf}}
